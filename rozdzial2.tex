\section{Skutki i Prognozy}

\begin{frame}{Wpływ na rynek konsumencki i gamingowy}
    Wzrost cen RAM-u uderza bezpośrednio w portfele użytkowników:
    \begin{description}
        \item[Laptopy] Standardowe modele „biurowe” z 16GB RAM podrożały średnio o 450 PLN w porównaniu do 2024 roku.
        \item[Gaming] Zestawy 32GB DDR5-6000, które kosztowały ok. 500 PLN, obecnie osiągają ceny rzędu 950-1100 PLN.
        \item[Smartfony] Flagowce oferują mniej pamięci (powrót do 8GB zamiast 12GB), aby utrzymać dotychczasowe ceny urządzeń.
    \end{description}
\end{frame}

\begin{frame}{Analiza trendów cenowych (Prognoza na 2026)}
    Prognozowany przebieg cen kontraktowych w ujęciu kwartalnym:
    \begin{table}
        \centering
        \small
        \begin{tabular}{@{}lll@{}}
            \toprule
            \textbf{Okres} & \textbf{Trend} & \textbf{Zmiana \% (r/r)} \\ \midrule
            Q1 2026 & Gwałtowny wzrost (Szczyt) & +55\% \\
            Q2 2026 & Stabilizacja na wysokim poziomie & +52\% \\
            Q3 2026 & Powolny trend spadkowy & +30\% \\
            Q4 2026 & Powrót do względnej normy & +10\% \\ \bottomrule
        \end{tabular}
    \end{table}
    \textit{Źródło: Opracowanie własne na podstawie raportów Gartner i TrendForce 2026.}
\end{frame}

\begin{frame}{Kiedy warto kupować?}
    \begin{block}{Rekomendacje zakupowe}
        \begin{itemize}
            \item \textbf{Styczeń - Czerwiec 2026:} Najgorszy czas na zakupy. Warto rozważyć zakup pamięci używanej.
            \item \textbf{Lipiec - Wrzesień 2026:} Pierwsze wyprzedaże magazynowe i spadek cen o ok. 15\%.
            \item \textbf{Październik 2026+:} Najlepszy moment na modernizację komputera (spodziewane nasycenie rynku).
        \end{itemize}
    \end{block}
    Dodatkowym czynnikiem stabilizującym będzie premiera nowych architektur CPU, które wymuszą na producentach zwiększenie podaży modułów DDR5.
\end{frame}