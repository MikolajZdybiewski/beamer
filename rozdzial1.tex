\section{Przyczyny wzrostu cen}

\begin{frame}{Kontekst rynkowy: Początek 2026 roku}
    Sytuacja na rynku półprzewodników stała się krytyczna w grudniu 2025 r.
    \begin{itemize}
        \item \textbf{Wyczerpanie zapasów:} Magazyny z tanimi modułami DDR4/DDR5 z lat 2023-2024 zostały całkowicie opróżnione.
        \item \textbf{Inflacja produkcyjna:} Koszty energii w fabrykach w Azji wzrosły o 15\%, co bezpośrednio przekłada się na cenę wafla krzemowego.
        \item \textbf{Efekt skali AI:} Popyt na serwerowe układy obliczeniowe wzrósł o 300\% w skali globalnej.
    \end{itemize}
\end{frame}

\begin{frame}{Efekt kanibalizacji produkcji (HBM vs DDR5)}
    Największym problemem jest ograniczona moc przerobowa fabryk (fabów).
    \begin{block}{Dlaczego brakuje miejsca?}
        Pamięci HBM3E/HBM4 (używane w kartach NVIDIA i AMD) wymagają znacznie bardziej skomplikowanego procesu pakowania. Jedna linia produkcyjna HBM zajmuje tyle miejsca, co trzy linie standardowej pamięci DDR5.
    \end{block}
    \begin{itemize}
        \item Producenci tacy jak Samsung i SK Hynix wybierają produkcję HBM, ponieważ marża jest tam o 400\% wyższa niż w przypadku RAM-u do domowych komputerów.
    \end{itemize}
\end{frame}

\begin{frame}{Problemy z łańcuchem dostaw}
    \begin{itemize}
        \item \textbf{Kryzys gazowy:} Ponowny wzrost cen neonu i ksenonu (niezbędnych do litografii) z powodu niepokojów geopolitycznych.
        \item \textbf{Logistyka:} Problemy z transportem morskim wydłużyły czas oczekiwania na dostawę gotowych modułów z 4 do 12 tygodni.
        \item \textbf{Monopolizacja:} Trzech graczy (Samsung, Micron, Hynix) kontroluje ponad 90\% rynku, co sprzyja agresywnej polityce cenowej.
    \end{itemize}
\end{frame}